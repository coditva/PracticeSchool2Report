% vi:ft=tex

\section{Final product}

\paragraph{} The final working prototype was able to submit a job, execute jobs,
list running jobs and show the result of the jobs successfully on a
\glssymbol{pbs} cluster. This prototype can be built with the latest Spark
(currently version \texttt{2.4.0}) and the latest Scala (currently version
\texttt{2.12.7}).


\section{Change for clients}

\paragraph{Job submission} The client see no change in submission of a
\gls{spark} job on a \glssymbol{pbs} cluster except adding \texttt{--master pbs}
as a command-line option while submission.

\paragraph{Job monitoring} The jobs can be seen on a webpage similar to
\gls{spark}'s Master UI webpage. The users have the ability to kill jobs from
this page itself.

\paragraph{Job result and logs} The job result and logs are currently accessed
as \glssymbol{pbs} job logs. This is a regression from what \gls{spark} users
are used to. But this issue can be solved when the Web user interface is
complete, which can fetch data from the job logs and display on a web page. This
behavior is similar to \gls{spark}'s way for displaying job logs.


\section{Issues and further scope of improvement}

\paragraph{Security} In the current prototype, any user who can access the web
UI is able to kill user jobs. This is because the kill request is sent as user
who started the UI. Since the UI is started on the server by the user with job
deletion rights and all kill requests are sent with that user, there is no way
of differentiating between different users.

\paragraph{Usability} The current prototype can be optimized to prevent
deadlocks in the cluster by allocating one big chunk of resources at the same
time rather than in smaller, per-executor chunks. Currently, the user also has
to manually \texttt{scp} the resources to the server when submitting a job. This
can be improved by using the stage-in utility of \glssymbol{pbs}.

\paragraph{User Interface} The user interface of the current prototype can be
improved to provide an interface very similar to \gls{spark} with application
logs and such. There is a work-in-progress pull request on the prototype
directory which will resolves this issue.

\paragraph{} More issues can be found in the issues tab of the GitHub repository
on the prototype. See references.
