% vi:ft=tex

\section{About Apache Spark?}

\paragraph{} \Gls{spark} is the leading frontrunner in Big Data Analytics and
Machine Learning. A \gls{spark} application can run for days crunching on data
and churning out results. This is why most \gls{spark} applications utilize a
cluster of nodes to execute tasks to decrease the time. A \gls{spark} cluster
can be set up using its own cluster manager known as \gls{spark} Standalone
Cluster Manager or one of third party cluster managers like \gls{yarn},
\gls{mesos} or \gls{kubernetes}.

\paragraph{} Please see \hyperref[sec:appendix-spark]{Appendix for \gls{spark}}
for more information on \gls{spark}'s architecture.


\section{About \glssymbol{pbs}}

\paragraph{} \glssymbol{pbs} is a workload management system which optimizes
\gls{job} scheduling in \gls{hpc} environments --- clusters, clouds and super
computers. A \glssymbol{pbs} job can be anything from a batch script to a C/C++
application. It is an open sourced application with commercial support also
available which can be run on Linux or Windows platforms.

\paragraph{} Please see \hyperref[sec:appendix-pbs]{Appendix for
\glssymbol{pbs}} for more information on \glssymbol{pbs}'s architecture.


\section{Need for adding \glssymbol{pbs} as a cluster manager in \gls{spark}}

\paragraph{} There are many organizations which require running \gls{hpc} jobs
as well as Big Data jobs on a cluster. But none of the cluster managers which
integrate with \gls{spark} are capable of running an \gls{hpc} job, thus
creating a need to add \gls{pbs} as a pluggable scheduler in \gls{spark}. This
project aims to do just that.
